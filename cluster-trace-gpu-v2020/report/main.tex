\documentclass{report}
\usepackage[utf8]{inputenc}
\usepackage{geometry}
 \geometry{
 a4paper,
 total={170mm,257mm},
 left=20mm,
 top=20mm,
 }
 \usepackage{graphicx}
%  \usepackage{titling}

\title{Adaptation Approach}
\author{Christian Bauer}
% \date{November 2022}
 
 \usepackage{fancyhdr}

\begin{document}

  \maketitle

  \tableofcontents

  \section*{Abstract}

  \chapter*{Publication of the Thesis}

    \section*{Big Data Pipeline Scheduling and Adaptation on the Computing Continuum}

  \chapter{Introduction}

    \section{Motivation and Scope}
    \section{Research Problems}
    \section{Research Objectives}
    \section{Research Methodology}
    \section{Thesis Outline}

  \chapter{State of the Art}

    Go based on the objectives or machine learning methods, research prediction....
    Categorization of state of the art, how many papers are there for cloud, fog, or for machine learning stuff.

    \section{Long-Short Term Memory}
    % \section{Alibaba/Google Resource Analysis}
    \section{Public Cloud Provider Traces in Available Data}
    \section{Resource Prediction based on Machine Learning}
    % what are the works that focus on the processing and tracing of resource utilisation
  
  \chapter{Model} % is the methodology

    \section{Application Model}
    \section{Resource Model}
    \section{Network Model}
    \section{Monitoring Model}
    \section{Resource Prediction Model}

  \chapter{Architecture and Implementation}

    \section{Architecture of the Software}

      \subsection{Adaptation Loop}

    \section{Monitoring}

    \section{Data Analysis and Preprocessing}

      \subsection{Alibaba Resource Analysis}
      \subsection{LSTM Architecture}

    \section{Adaptation}
      \subsection{Resource Prediction}
      % flow chart of resource prediction
      \subsection{DataFrame Scaler}
      \subsection{Penalty Mean Squared Error Loss Function}
    \section{Conclusion}

  \chapter{Evaluation and Results}

    % \section{Introduction}
    \section{Evaluation Setup}
      How everything was set up (kubernetes, ml,...)
      \subsection{Kubernetes}
      \subsection{NetData}
      \subsection{Prometheus}
      \subsection{LSTM Model Setup}
      \subsection{Metrics}
      \subsection{Weights \& Biases}
    \section{Evaluation Scenario}
    % the different evaluations I did like task type or batch size
    \section{Monitoring}
    \section{Data Analysis}
    \section{Adaptation}
    \section{Conclusion}

  \chapter{Conclusions and Future Work}

    \section{Conclusions}
    \section{Future Work}

  \chapter{Reference}






\end{document}

